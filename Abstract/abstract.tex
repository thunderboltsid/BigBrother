\documentclass[paper=a4, fontsize=12pt]{article}
\usepackage[margin=1.2in]{geometry}
\usepackage[T1]{fontenc}
\usepackage{fourier}
\usepackage{hyperref}
\usepackage[english]{babel}
\usepackage[protrusion=true,expansion=true]{microtype}				
\usepackage{amsmath,amsfonts,amsthm}									
\usepackage{graphicx}
\usepackage{natbib}											
\usepackage{sectsty}												
\allsectionsfont{\centering \normalfont\scshape}
\usepackage{fancyhdr}
\pagestyle{fancyplain}
\fancyfoot[C]{Prof. Dr. Adalbert F.X. Wilhelm\\JTBU-020003 $:$ Big Data: Big Boon and Big Brother}

\newcommand{\horrule}[1]{\rule{\linewidth}{#1}} 	
\title{
		\vspace{-1in} 	
		\usefont{OT1}{bch}{b}{n}
		\normalfont \normalsize \textsc{School of Humanities and Social Sciences\\Jacobs University Bremen} \\ [25pt]
		\horrule{0.5pt} \\[0.4cm]
		\huge Debating the ethical and technological challenges surrounding open access to clinical trial data\\
		\horrule{2pt} \\[0.5cm]
}
\author{Nick Lee\\Rashed Popalzai\\Siddharth Shukla\\Trang Le\\\\\today}
\date{}
\begin{document}
\maketitle
\newpage
\fancyhead[C]{Nick Lee $\cdot$ Rashed Popalzai $\cdot$ Siddharth Shukla $\cdot$ Trang Le $\cdot$ Jacobs University Bremen}

\section*{Abstract}
Clinical trial studies are systemic, complex and costly sets of experiments that investigate whether specific pharmaceuticals, medical treatment strategies, medical devices are safe and effective for human use.\\\\
In the past few decades, clinical trial study results have been growing not only in volume but also in diversity and in incongruity. The sharing of such valuable data can enhance the understanding of the complex medical problem by combining multiple study results and perform meta-analysis to extend scientific findings beyond the original theory. Even incomplete or failed clinical trial can be extremely useful in avoiding similar mistakes. In order for the clinical trial data to be publicly available for the investigators, Networks need to be established and databases administered.\\\\
The Clinical Research Network (CRN) is, for example, designed to facilitate data sharing between the interested stakeholders. In the US, clinicaltrials.gov was developed by the  National Institute of Health (NIH) as a platform for the investigators to post their trial findings and for the public and other interested stakeholders to look into the data. Besides the obvious potential knowledge can be withdrawn from such valuable information, the problems associated with data sharing and the complexity of data management are not to be underestimated.\\\\
On this project we will try to look at the state of the clinical data network and try to answer some ethical questions regarding the morality of sharing the findings of the trials and how can it affect the individuals and groups participating in the studies. We will also talk about the clinical trials in the US and will try to answer how costly and effective it is to share clinical trial data and make it available to the public.

\end{document}